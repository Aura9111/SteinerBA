% !TeX root = ../main.tex
% Add the above to each chapter to make compiling the PDF easier in some editors.

\chapter{Summary}\label{chapter:conclusion}

We started by defining our problem setting, which was a procurement auction with single minded sellers and ourself in the position of an auctioneer trying to procure the minimum cost network that spans a set of required nodes. We then implemented two loss contraction algorithms for this problem setting, which are based on Steiner tree approximation algorithms. Since they both relied on a different algorithm for an initial Steiner tree we provided this algorithm as well. We used the Steiner tree problems at SteinLib \cite{Dui93} as testcases for our algorithms. The simplest one was the MST-approximation algorithm, which was used as the initial working tree for the other two algorithms. While its solution trees for our testcases all had decent average total tree costs, it was bested by both other algorithms when we compare the average total tree cost and the performance ratio. We then looked at the approximation algorithm presented by Berman and Ramaiyer \cite{BeRa94}, which provided slightly better average tree costs and a significantly better performance ratio, by including Steiner nodes, that appeared in subsets of terminals, into the final solution tree. The last algorithm we looked at was the $IRGH$-algorithm by Hougardy and Proemel \cite{HoPr99}, which improved the performance ratio once again. Its average tree cost was also better compared to the other two algorithms, but not consistently. In most cases it managed to outperform the other algorithms with its heuristic that helped include more Steiner nodes into the tree, allowing it to find more complex solutions, but in some cases this aggressive inclusion of Steiner nodes led to it increasing upon the cost of the initial MST-approximation. We then proceded to discern whether the algorithms in question would implement strategy proofness or incentive compatibility. This incentive compatibility is required to guarantee that the dominant strategy for our single-minded sellers is to report their true valuations for the edges they're selling to us. In order for the algorithms to be strategy proof they had to implement monotonicity and had to pay out critical payments to the winning seller. This is due to a Lemma proposed by Blumrosen and Nisan \cite{BlNi07}. We were able to disprove all three algorithms we implemented, since each algorithm has one or more rare cases, where monotonicity is violated. Despite these cases a cost reduction for an edge still increases the likelyhood of it appearing in the tree. Our algorithms not being monotonic did not prove a problem for computing the critical payments. We computed them using exponential search and binary search successively. The results of this binary search as well as the cost of the approximated solution for our three implemented algorithms can be referenced in Appendix \ref{chapter:appendix}. The lowest average tree cost,as well as the lowest performance ratio within our examples was achieved by the $IRGH$ algorithm, but since it is not strategy proof it is not reliably usable as a solution for our proposed setting. This is also true for the algorithm by Berman and Ramaiyer, as well as for the MST-approximation.