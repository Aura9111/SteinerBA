% !TeX root = ../main.tex
% Add the above to each chapter to make compiling the PDF easier in some editors.

\chapter{Introduction}\label{chapter:introduction}

There are a lot of problems that modern companies face which at least partly include network procurement problems. These problems are about finding the most efficient way to form a network. The input for these types of problems usually involves a number of points that need to be connected to each other within the desired network and a way to obtain connections between two points for a cost. This cost could be valued using money, time or opportunity cost, etc. There may be other criteria that the network has to fulfill, but we are going to focus on the foundations. The goal in any case is to find the minimum cost network fulfilling the desired criteria and connecting the required points. 
Some real-world examples could include networks, that span populated areas to provide e.g. water access, transport, telecommunication, electricity, etc. Other examples could include optimization problems like logistics or project planning. Even assignment problems can be modelled in this way by using a flow graph and assigning a opportunity cost to every decision. 
We are going to propose a very general setting for these network procurement problems, which is based on the Steiner tree problem. Since this problem is NP-complete we are going to look at approximation algorithms for our setting. We are going to implement two loss-contracting algorithms originally designed for the Steiner tree problem. We are going to use examples of Steiner tree problems on SteinLib \cite{Dui93} as testcases and we are going to use the reported optimal solution cost there to contrast the results of the two implemented algorithms. We are also going to check the strategy proofness of our algorithms in the proposed settings, which ensures that the valuation of the connections offered to us are authentic.  