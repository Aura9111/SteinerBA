\chapter{\abstractname}

%For this bachelor's thesis I'm going to implement two Approximation Algorithms for the Steiner Tree Problem and compute the critical payment for every edge in the graph using simple binary search. In order to compute these critical payments, I'm going to first prove that my graphs and its edge costs implement monotonicity.  Afterwards I'm going to compare the results of the different Approximation Algorithms. 

In economics there is an interest to find incentive compatible and efficiently computable auction mechanisms that give good approximations for optimal social welfare. This thesis will try to find such an approximation by using algorithms for the Steiner Tree Problem on Procurement Auctions.% that can be modelled using a graph like f. ex. transportation routes or network paths. 

The environment for the Steiner Tree Problem is a weighted, undirected graph G = (V, E), where V is the set of nodes and E is the set of edges within G. There is also a subset T $\subseteq$ V called "terminals" and a cost function d: E $\to \mathbb{R}$, that returns the added path cost for a given set of edges. The Steiner Problem now asks for the minimal tree connecting all terminals. This problem is NP-complete and the algorithms presented in this thesis give approximations for the solution. The first approximation algorithm we're going to look at was presented in a paper by Berman and Ramaiyer \cite{BeRa94} and constructs Steiner minimal trees for subsets of T with at most k elements,  adding them to the solution greedily and finally removing the redundant edges. It was proven that increasing the value of  parameter k, improves the approximation rate, which converges to 1.746 for k $\to \infty$. The second algorithm from Hougardy and Proemel \cite{HoPr99} improves the previous approximation rates by using a generalized version of the parameterized relative greedy heuristic (RGH) by Karpinski and Zelikovsky \cite{KaZe97} and iteratively applying it with different parameters $\alpha_i$ (i $\in$ [1, k]) to the previous iteration's output. They achieved the approximation ratio of 1.598 after 11 iterations and indicated the limit at 1.588 for k $\to \infty$. 
In a 2007 paper, Blumrosen and Nisan \cite{BlNi07} proved that a mechanism for single-minded bidders is incentive compatible iff it satisfies monotonicity and critical payment. We will therefore assess monotonicity for both implementations and afterwards compute critical payments. To find the critical payment for edge $e$ we will change the cost of $e$ in the input graph and check whether or not it is still included in the resulting Steiner Tree after applying the algorithm on this changed input graph. Using binary search we are going to find the maximum prize at which $e$ is still part of the tree, which is therefore the critical payment. These critical payments should vary for both algorithms as well as the total edge cost and the runtime. To conclude this thesis we will analyse these statistics and compare them between the two algorithms.

After implementing both algorithms \cite{githublink} and checking them for incentive compatibility it turned out, that Hougardy and Proemel's algorithm does not fulfill monotonicity and is therefore not strategy proof. Berman and Ramaiyer's algorithm is incentive compatible and was chosen as the best solution for our proposed setting.